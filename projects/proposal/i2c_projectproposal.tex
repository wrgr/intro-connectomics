% !TEX TS-program = pdflatex
% !TEX encoding = UTF-8 Unicode

% This is a simple template for a LaTeX document using the "article" class.
% See "book", "report", "letter" for other types of document.

\documentclass[11pt]{article} % use larger type; default would be 10pt

\usepackage[utf8]{inputenc} % set input encoding (not needed with XeLaTeX)


%%% PAGE DIMENSIONS
\usepackage{geometry} % to change the page dimensions
\geometry{a4paper} % or letterpaper (US) or a5paper or....


\usepackage{graphicx} % support the \includegraphics command and options

% \usepackage[parfill]{parskip} % Activate to begin paragraphs with an empty line rather than an indent

%%% PACKAGES
\usepackage{booktabs} % for much better looking tables
\usepackage{array} % for better arrays (eg matrices) in maths
\usepackage{paralist} % very flexible & customisable lists (eg. enumerate/itemize, etc.)
\usepackage{verbatim} % adds environment for commenting out blocks of text & for better verbatim
\usepackage{subfig} % make it possible to include more than one captioned figure/table in a single float
\usepackage{cite}
% These packages are all incorporated in the memoir class to one degree or another...

%%% HEADERS & FOOTERS
\usepackage{fancyhdr} % This should be set AFTER setting up the page geometry
\pagestyle{fancy} % options: empty , plain , fancy
\renewcommand{\headrulewidth}{0pt} % customise the layout...
\lhead{}\chead{}\rhead{}
\lfoot{}\cfoot{\thepage}\rfoot{}

%%% SECTION TITLE APPEARANCE
\usepackage{sectsty}
\allsectionsfont{\sffamily\mdseries\upshape} % (See the fntguide.pdf for font help)
% (This matches ConTeXt defaults)

%%% ToC (table of contents) APPEARANCE
\usepackage[nottoc,notlof,notlot]{tocbibind} % Put the bibliography in the ToC
\usepackage[titles,subfigure]{tocloft} % Alter the style of the Table of Contents
\renewcommand{\cftsecfont}{\rmfamily\mdseries\upshape}
\renewcommand{\cftsecpagefont}{\rmfamily\mdseries\upshape} % No bold!



\title{Mapping the Brain: An Introduction to Connectomics\\Your Project Title}
\author{Your name}
%\date{} % Activate to display a given date or no date (if empty),
         % otherwise the current date is printed 

\begin{document}
\maketitle

\section{Introduction}

Description of what your project is. Perhaps start off with an overview of the area of interest \textbackslash [and cite it, of course]. It is good practice to use the OCAR \cite{noel2013} technique which we will discuss in class: after you introduce your topic explain why this work is difficult, what you wish to do to overcome this, and why it has yet to be done. Finally, summarize why your progress on this project will be meaningful. This shouldn't be much longer than 5 or 6 sentences.

\section{Project Outline}
\subsection{Methods}
Describe where you are starting your project (based on what given results, procedures, knowledge, etc.) and where you plan to take it. Try to be clear as possible. It is not necessary for this to be exactly what you end up doing, but it should be pretty close. This, potentially more technical, section should link back to your introduction (i.e. the opportunities, challenges, action, and resolution are all described in more detail in terms of the 'how' instead of the 'what', as above).
\subsection{Schedule}
Since this is a short course, it is advised that you and your group make a schedule of minor and major deadlines required throughout the project so that you can ensure that you have enough time to complete your project. Making a graphical timeline may be a nice idea to help visualize this and divide up tasks. 
\subsection{Allocation of Tasks}
To further identify what needs to be done, and how it will be done, it can be useful to identify who will be performing each task. It is of course alright to have multiple people working on each part, or all members do everthing together. This will also help ensure group member accountability.
\bibliography{yourbibname}{}
\bibliographystyle{plain}

\end{document}

























